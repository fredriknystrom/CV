%%%%%%%%%%%%%%%%%
% This is an example CV created using altacv.cls (v1.1.5, 1 December 2018) written by
% LianTze Lim (liantze@gmail.com), based on the
% Cv created by BusinessInsider at http://www.businessinsider.my/a-sample-resume-for-marissa-mayer-2016-7/?r=US&IR=T
%
%% It may be distributed and/or modified under the
%% conditions of the LaTeX Project Public License, either version 1.3
%% of this license or (at your option) any later version.
%% The latest version of this license is in
%%    http://www.latex-project.org/lppl.txt
%% and version 1.3 or later is part of all distributions of LaTeX
%% version 2003/12/01 or later.
%%%%%%%%%%%%%%%%

%% If you are using \orcid or academicons
%% icons, make sure you have the academicons
%% option here, and compile with XeLaTeX
%% or LuaLaTeX.
% \documentclass[10pt,a4paper,academicons]{altacv}

%% Use the "normalphoto" option if you want a normal photo instead of cropped to a circle
%\documentclass[10pt,a4paper,normalphoto]{altacv}

\documentclass[10pt,a4paper,ragged2e]{altacv}

%% AltaCV uses the fontawesome and academicon fonts
%% and packages.
%% See texdoc.net/pkg/fontawecome and http://texdoc.net/pkg/academicons for full list of symbols. You MUST compile with XeLaTeX or LuaLaTeX if you want to use academicons.

% Change the page layout if you need to
\geometry{left=1.5cm,right=9.5cm,marginparwidth=6.8cm,marginparsep=1.2cm,top=1cm,bottom=1.25cm}

% Change the font if you want to, depending on whether
% you're using pdflatex or xelatex/lualatex
\ifxetexorluatex
  % If using xelatex or lualatex:
  \setmainfont{Carlito}
\else
  % If using pdflatex:
  \usepackage[utf8]{inputenc}
  \usepackage[T1]{fontenc}
  \usepackage[default]{lato}
  \usepackage{hyperref}
\fi


% Change the colours if you want to
\definecolor{Grey}{HTML}{34495E}
\definecolor{Black}{HTML}{2E2E2E} %2E2E2E
\definecolor{Green}{HTML}{3CB043}
\definecolor{Red}{HTML}{960A0A}
\definecolor{Blue}{HTML}{0077B5}

\colorlet{Green}{Green}
\colorlet{Red}{Red}
\colorlet{Blue}{Blue}
\colorlet{heading}{Grey}
\colorlet{accent}{Black}
\colorlet{emphasis}{Black}
\colorlet{body}{Black}


% Change the bullets for itemize and rating marker
% for \cvskill if you want to
\renewcommand{\itemmarker}{{\small\textbullet}}
\renewcommand{\ratingmarker}{\faCircle}

%% sample.bib contains your publications
\addbibresource{sample.bib}

\begin{document}

\name{Resumé - Fredrik Nyström}
\tagline{Student at the Computer Science division at Chalmers University of Technology}
% Cropped to square from https://en.wikipedia.org/wiki/Marissa_Mayer#/media/File:Marissa_Mayer_May_2014_(cropped).jpg, CC-BY 2.0
\photo{4cm}{profile.jpg}
\personalinfo{%
  % Not all of these are required!
  % You can add your own with \printinfo{symbol}{detail}
  \email{frediknystroms@gmail.com}
  \phone{+46 709 15 16 06}
%  \mailaddress{Address, Street, 00000 County}
  \location{Gothenburg, Sweden}
%  \homepage{marissamayr.tumblr.com/}
%  \twitter{@marissamayer}
\\
   \github{\href{https://github.com/fredriknystrom
  }{github.com/fredriknystrom}}
   \linkedin{\href{www.linkedin.com/in/fredrik-nyström-b1757322b/}{linkedin.com/in/fredrik-nyström-b1757322b}}
}


%% Make the header extend all the way to the right, if you want.
\begin{fullwidth}
\makecvheader
\end{fullwidth}

%% Depending on your tastes, you may want to make fonts of itemize environments slightly smaller
\AtBeginEnvironment{itemize}{\small}

%% Provide the file name containing the sidebar contents as an optional parameter to \cvsection.
%% You can always just use \marginpar{...} if you do
%% not need to align the top of the contents to any
%% \cvsection title in the "main" bar.

\cvsection[page1sidebar]{Education}
\cvevent{Master of Science in Computer Science and Engineering}{Chalmers University of Technology}{aug 2020 -- ongoing}{Gothenburg, Sweden}
\begin{itemize}
    \item Estimated year of examination 2025
\end{itemize}

\divider

\cvevent{Naturvetenskapsprogrammet}{Sundsgymnasiet}{2017 -- 2020}{Vellinge, Sweden}

\cvsection{Work Experience}

\cvevent{Software developer}{Axis Communications AB}{june 2022 – aug 2022}{Lund, Sweden}
\begin{itemize}
    \item Developed a graphical user interface with \textbf{Python}, \textbf{PyQt5} and \textbf{CSS} to ease testing of camera lenses. The
\end{itemize}

\divider

\cvevent{Extra teacher}{Studdybuddy}{jan 2021 -- june 2022}{Online}
    \begin{itemize}
    \item Undervisade gymnasieelever digitalt i matte och programmering.
\end{itemize}

\divider

\cvevent{Maskinoperatör}{ÅR Carton}{2019, 2020}{Lund, Sverige}
\begin{itemize}
    \item Under sommrarna 2019 och 2020 arbetade jag som maskinoperatör vid produktionslinjen.
\end{itemize}

\newpage

% Begin full width on the next page (where only projects are shown)
\begin{fullwidth}
\cvsection{Projekt}

\cvevent{Larmsystem}{C, Latex, GitHub}{}{}
\begin{itemize}
  \item I kursen DAT290 var jag projektledare vid utveckling av ett larmsystem bestående av en enchipsdator.
  \item Systemet utvecklades i \textbf{C}. För dokumentation användes \textbf{Latex}. För versionskontroll användes \textbf{Github}.
\end{itemize}

\divider

\cvevent{\href{https://github.com/fredriknystrom/Pong}{Pong i pygame}}{Python}{}{}
\begin{itemize}
  \item Projektet utvecklades för att förstå och visualisera hitbox och rörelse av olika object.
  \item Projektet implementerades i \textbf{Python} och pygame användes för grafik.
\end{itemize}

\divider

\cvevent{\href{https://github.com/fredriknystrom/FIFAWorldCupTip}{VM tips}}{Python}{}{}
\begin{itemize}
  \item Genererar ett automatiserat excelark från python kod. Användaren fyller endast i resultat för varje match så skickas lagen automatiskt vidare genom slutspelet. Tanken är att det ska skapas ett automatiskt rättnings script till detta senare.
  \item Projektet implementerades i \textbf{Python} och biblioteket openpyxl användes.
\end{itemize}

% Other merits
\cvsection{Other merits}

\cvevent{Datateknologsektionens idrottskommitté, iDrott}{}{maj 2022 -- pågående}{}
\begin{itemize}
  \item Jag är aktiv i planeringen av och arrangerar veckovis idrottsträningar, för att uppmuntra  stillasittande datateknologer till fysisk aktivitet. 
  \item Jag är lagledare för datateknologsektionens fotbollslag.
\end{itemize}

\divider

\cvevent{Ordförande Brf Volrat Thams styrelse}{}{maj 2022 -- pågående}{}
\begin{itemize}
  \item Arbetet som ordförande innefattar utskick av kallelse och dagordning till styrelesemöte varje månad. Beslut gällande omläggning av lån. Attestering av månadsavier och leverantörsfakturor.
\end{itemize}

\divider

\cvevent{Svensk Klassiker}{}{2022/2023}{}

\divider

\cvevent{Adlerbertska Stiftelsernas stipendium}{}{2021}{}

\divider

\cvevent{B-Körtkort, truckkort A2 och A4}{}{2019}{}

% References moved from page 1
\cvsection{Referenser}
Referenser/intyg/betyg medtages vid intervjutillfälle eller skickas in vid förfrågan.

% End fullwidth for second page
\end{fullwidth}

\end{document}
