%%%%%%%%%%%%%%%%%
% This is an example CV created using altacv.cls (v1.1.5, 1 December 2018) written by
% LianTze Lim (liantze@gmail.com), based on the
% Cv created by BusinessInsider at http://www.businessinsider.my/a-sample-resume-for-marissa-mayer-2016-7/?r=US&IR=T
%
%% It may be distributed and/or modified under the
%% conditions of the LaTeX Project Public License, either version 1.3
%% of this license or (at your option) any later version.
%% The latest version of this license is in
%%    http://www.latex-project.org/lppl.txt
%% and version 1.3 or later is part of all distributions of LaTeX
%% version 2003/12/01 or later.
%%%%%%%%%%%%%%%%

%% If you are using \orcid or academicons
%% icons, make sure you have the academicons
%% option here, and compile with XeLaTeX
%% or LuaLaTeX.
% \documentclass[10pt,a4paper,academicons]{altacv}

%% Use the "normalphoto" option if you want a normal photo instead of cropped to a circle
%\documentclass[10pt,a4paper,normalphoto]{altacv}

\documentclass[10pt,a4paper,ragged2e]{altacv}

%% AltaCV uses the fontawesome and academicon fonts
%% and packages.
%% See texdoc.net/pkg/fontawecome and http://texdoc.net/pkg/academicons for full list of symbols. You MUST compile with XeLaTeX or LuaLaTeX if you want to use academicons.

% Change the page layout if you need to
\geometry{left=1.5cm,right=9.5cm,marginparwidth=6.8cm,marginparsep=1.2cm,top=1cm,bottom=1.25cm}

% Change the font if you want to, depending on whether
% you're using pdflatex or xelatex/lualatex
\ifxetexorluatex
  % If using xelatex or lualatex:
  \setmainfont{Carlito}
\else
  % If using pdflatex:
  \usepackage[utf8]{inputenc}
  \usepackage[T1]{fontenc}
  \usepackage[default]{lato}
  \usepackage{hyperref}
\fi


% Change the colours if you want to
\definecolor{Grey}{HTML}{34495E}
\definecolor{Black}{HTML}{2E2E2E} %2E2E2E
\definecolor{Green}{HTML}{3CB043}
\definecolor{Red}{HTML}{960A0A}
\definecolor{Blue}{HTML}{0077B5}

\colorlet{Green}{Green}
\colorlet{Red}{Red}
\colorlet{Blue}{Blue}
\colorlet{heading}{Grey}
\colorlet{accent}{Black}
\colorlet{emphasis}{Black}
\colorlet{body}{Black}


% Change the bullets for itemize and rating marker
% for \cvskill if you want to
\renewcommand{\itemmarker}{{\small\textbullet}}
\renewcommand{\ratingmarker}{\faCircle}

%% sample.bib contains your publications
\addbibresource{sample.bib}

\begin{document}

\name{Resumé - Fredrik Nyström}
\tagline{Student at the Computer Science division at Chalmers University of Technology}
% Cropped to square from https://en.wikipedia.org/wiki/Marissa_Mayer#/media/File:Marissa_Mayer_May_2014_(cropped).jpg, CC-BY 2.0
\photo{4cm}{profile.jpg}
\personalinfo{%
  % Not all of these are required!
  % You can add your own with \printinfo{symbol}{detail}
  \email{frediknystroms@gmail.com}
  \phone{+46 709 15 16 06}
%  \mailaddress{Address, Street, 00000 County}
  \location{Gothenburg, Sweden}
%  \homepage{marissamayr.tumblr.com/}
%  \twitter{@marissamayer}
\\
   \github{\href{https://github.com/fredriknystrom}{github.com/fredriknystrom}}
   \linkedin{\href{www.linkedin.com/in/fredrik-nyström-b1757322b/}{linkedin.com/in/fredrik-nyström-b1757322b}}
}


%% Make the header extend all the way to the right, if you want.
\begin{fullwidth}
\makecvheader
\end{fullwidth}

%% Depending on your tastes, you may want to make fonts of itemize environments slightly smaller
\AtBeginEnvironment{itemize}{\small}

%% Provide the file name containing the sidebar contents as an optional parameter to \cvsection.
%% You can always just use \marginpar{...} if you do
%% not need to align the top of the contents to any
%% \cvsection title in the "main" bar.

\cvsection[page1sidebar]{Education}
\cvevent{Master of Science in Computer Science and Engineering}{Chalmers University of Technology}{aug 2020 -- ongoing}{Gothenburg, Sweden}
\begin{itemize}
    \item Estimated year of examination 2025
\end{itemize}

\divider

\cvevent{Natural Science Program}{Sundsgymnasiet}{2017 - 2020}{Vellinge, Sweden}

\cvsection{Work Experience}

\cvevent{Teaching assistant}{Chalmers University of Technology}{oct 2022 - ongoing}{Gothenburg, Sweden}
\begin{itemize}
    \item I am currently working as a supervisor during exercise and lab sessions for 350+ students in \textbf{Introduction to computer engineering}. 
    Among the topics covered in the course are assembly programming, Boolean algebra, circuit design and hexadecimal 
    number representation.
\end{itemize}

\divider

\cvevent{Software developer}{Axis Communications AB}{june 2022 – aug 2022}{Lund, Sweden}
\begin{itemize}
    \item I developed a gui using \textbf{Python}, \textbf{PyQt5} and \textbf{CSS} to streamline selection of settings
    for a machine that tests camera lenses. The development was performed in a group of two and we spent most of time 
    doing pair programming which was a new experience for me. We had weekly meetings with our manager where we presented
    our progress and recieved feedback. Except for the meetings the work was performed independently and I grew 
    a lot both at personal level and as a programmer. Axis were very happy with our result.
\end{itemize}

\divider

\cvevent{Extra teacher}{Studdybuddy}{jan 2021 - june 2022}{Online}
    \begin{itemize}
    \item I taught math and programming to high school students. 
    I developed great patience and learnt how to explain a concept in different ways.
\end{itemize}

\divider

\cvevent{Machine operator}{ÅR Carton}{2019, 2020}{Lund, Sweden}
\begin{itemize}
    \item During the summer 2019 and 2020 I worked as a machine operator at the production 
    line.
\end{itemize}

\newpage

% Begin full width on the next page (where only projects are shown)
\begin{fullwidth}
\cvsection{Projects}

\cvevent{Alarm system}{C, Latex, GitHub}{}{}
\begin{itemize}
  \item In the course DAT290 I was the project leader during the development of an alarmsystem 
  consisting of a single chip computer. Besides wrtitng code in \textbf{C} I made sure that everyone 
  knew what to work on and prioritize. I arranged weekly meetings where we discussed our progress. 
  The result was good and I recieved the highest grade.
  \item The system was developed in \textbf{C} and \textbf{Github} was used for versioncontrol.
  \textbf{Latex} was used as documentation tool.
   
\end{itemize}

\divider

\cvevent{\href{https://github.com/fredriknystrom/worldcup}{World Cup Tip}}{Python, Excel, HTML, CSS}{}{}
\begin{itemize}
  \item The first part of the project generates an automated excel sheet created in \textbf{Python} 
  making use of the \textbf{openpyxl} library. The sheet contains groupmatches where the user has 
  to fill in the result of each match. The winning teams from the group will then automitically be filled in 
  at matches in the playoff bracket.
  \item The second part of the project is a script that reads excel files and compares them
  against a file with the correct answers. The score for each tip is stored in a text file.
  \item The last part of the project is a simple website to show the scoreboard of the participants in the tip.
   The website was created in \textbf{HTML} and \textbf{CSS}.
\end{itemize}

% Other merits
\cvsection{Other merits}

\cvevent{Computer Science sportscommittee, iDrott}{}{may 2022 -- ongoing}{}
\begin{itemize}
  \item I plan and arrange weekly sports to encourge computer science students
   to perform physical activities. 
  \item I am the team leader for Computer Science soccer team.
\end{itemize}

\divider

\cvevent{Chairman for Housing cooperative Volrat Tham's board}{}{maj 2022 -- ongoing}{}
\begin{itemize}
  \item The work contains monthly boardmeetings where make decision regarding new members, 
  loans and discuss wheter we any repairs necessary for the building. As Chairman I also certify 
  supplier invoices and prepare for the meetings.
\end{itemize}

\divider

\cvevent{Svensk klassiker}{}{2022/2023}{}

\divider

\cvevent{Adlerbertska Stiftelsernas scholarship}{}{2021, 2022}{}

\divider

\cvevent{Driver's license}{}{2019}{}

% References moved from page 1
\cvsection{References}
References/certificates/grades are brought at interview opportunity or will be sent at request.

% End fullwidth for second page
\end{fullwidth}

\end{document}
