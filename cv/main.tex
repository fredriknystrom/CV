%%%%%%%%%%%%%%%%%
% This is an example CV created using altacv.cls (v1.1.5, 1 December 2018) written by
% LianTze Lim (liantze@gmail.com), based on the
% Cv created by BusinessInsider at http://www.businessinsider.my/a-sample-resume-for-marissa-mayer-2016-7/?r=US&IR=T
%
%% It may be distributed and/or modified under the
%% conditions of the LaTeX Project Public License, either version 1.3
%% of this license or (at your option) any later version.
%% The latest version of this license is in
%%    http://www.latex-project.org/lppl.txt
%% and version 1.3 or later is part of all distributions of LaTeX
%% version 2003/12/01 or later.
%%%%%%%%%%%%%%%%

%% If you are using \orcid or academicons
%% icons, make sure you have the academicons
%% option here, and compile with XeLaTeX
%% or LuaLaTeX.
% \documentclass[10pt,a4paper,academicons]{altacv}

%% Use the "normalphoto" option if you want a normal photo instead of cropped to a circle
%\documentclass[10pt,a4paper,normalphoto]{altacv}

\documentclass[10pt,a4paper,ragged2e]{altacv}

%% AltaCV uses the fontawesome and academicon fonts
%% and packages.
%% See texdoc.net/pkg/fontawecome and http://texdoc.net/pkg/academicons for full list of symbols. You MUST compile with XeLaTeX or LuaLaTeX if you want to use academicons.

% Change the page layout if you need to
\geometry{left=1.5cm,right=9.5cm,marginparwidth=6.8cm,marginparsep=1.2cm,top=1cm,bottom=1.25cm}

% Change the font if you want to, depending on whether
% you're using pdflatex or xelatex/lualatex
\ifxetexorluatex
  % If using xelatex or lualatex:
  \setmainfont{Carlito}
\else
  % If using pdflatex:
  \usepackage[utf8]{inputenc}
  \usepackage[T1]{fontenc}
  \usepackage[default]{lato}
  \usepackage{hyperref}
\fi


% Change the colours if you want to
\definecolor{Grey}{HTML}{34495E}
\definecolor{Black}{HTML}{2E2E2E} %2E2E2E
\definecolor{Green}{HTML}{3CB043}
\definecolor{Red}{HTML}{960A0A}
\definecolor{Blue}{HTML}{0077B5}

\colorlet{Green}{Green}
\colorlet{Red}{Red}
\colorlet{Blue}{Blue}
\colorlet{heading}{Grey}
\colorlet{accent}{Black}
\colorlet{emphasis}{Black}
\colorlet{body}{Black}


% Change the bullets for itemize and rating marker
% for \cvskill if you want to
\renewcommand{\itemmarker}{{\small\textbullet}}
\renewcommand{\ratingmarker}{\faCircle}

%% sample.bib contains your publications
\addbibresource{sample.bib}

\begin{document}

\name{CV - Fredrik Nyström}
\tagline{Student på datateknikprogrammet på Chalmers tekniska högskola}
% Cropped to square from https://en.wikipedia.org/wiki/Marissa_Mayer#/media/File:Marissa_Mayer_May_2014_(cropped).jpg, CC-BY 2.0
\photo{4cm}{profile.jpg}
\personalinfo{%
  % Not all of these are required!
  % You can add your own with \printinfo{symbol}{detail}
  \email{frediknystroms@gmail.com}
  \phone{+46 709 15 16 06}
%  \mailaddress{Address, Street, 00000 County}
  \location{Göteborg, Sverige}
%  \homepage{marissamayr.tumblr.com/}
%  \twitter{@marissamayer}
\\
   \github{\href{https://github.com/fredriknystrom
  }{github.com/fredriknystrom}}
   \linkedin{\href{www.linkedin.com/in/fredrik-nyström-b1757322b/}{linkedin.com/in/fredrik-nyström-b1757322b}}
}


%% Make the header extend all the way to the right, if you want.
\begin{fullwidth}
\makecvheader
\end{fullwidth}

%% Depending on your tastes, you may want to make fonts of itemize environments slightly smaller
\AtBeginEnvironment{itemize}{\small}

%% Provide the file name containing the sidebar contents as an optional parameter to \cvsection.
%% You can always just use \marginpar{...} if you do
%% not need to align the top of the contents to any
%% \cvsection title in the "main" bar.

\cvsection[page1sidebar]{Utbildning}
\cvevent{Civilingenjör inom datateknik}{Chalmers tekniska högskola}{aug 2020 -- pågående}{Göteborg, Sverige}
\begin{itemize}
    \item Beräknad examen 2025
\end{itemize}

\divider

\cvevent{Naturvetenskapsprogrammet}{Sundsgymnasiet}{2017 -- 2020}{Vellinge, Sverige}

\cvsection{Arbetslivserfarenhet}

\cvevent{Amanuens}{Chalmers tekniska högskola}{okt 2022 - pågående}{Göteborg, Sverige}
\begin{itemize}
    \item Jag är handledare för övnings och laborationspass för 350+ studenter i \textbf{Grundläggande Datorteknik}. 
    Kursen innefattar bland annat assembler-programmering, Boolesk algebra och hexadecimal 
    nummerrepresentation.
\end{itemize}

\divider

\cvevent{Mjukvaruutvecklare}{Axis Communications AB}{juni 2022 – aug 2022}{Lund, Sverige}
\begin{itemize}
  \item Jag utvecklade ett gui med hjälp av \textbf{Python}, \textbf{PyQt5} och \textbf{CSS} för att enkelt kunna
  välja inställningar för testning av kameralinser till en maskin. Utvecklingen utfördes i en grupp om två och vi spenderade
  stora delar av tiden med att parprogramera vilket var en ny erfarenhet för mig. Vi hade veckovisa möten med vår chef
  där vi fick visa våra framsteg och få återkoppling. Förutom mötena arbetade vi mycket självständigt och jag växte 
  mycket både som person och som programmerare. Axis var mycket nöjda med vårt arbete.
\end{itemize}

\divider

\cvevent{Extralärare}{Studdybuddy}{jan 2021 -- juni 2022}{Online}
    \begin{itemize}
    \item Jag undervisade i matte och programmering för gymnasieelever. 
    Jag förbättrade mitt tålamod och fick öva mycket på hur man kan förklara 
    samma koncept på flera olika sätt.
\end{itemize}

\divider

\cvevent{Maskinoperatör}{ÅR Carton}{2019, 2020}{Lund, Sverige}
\begin{itemize}
    \item Under somrarna 2019 och 2020 arbetade jag som maskinoperatör vid produktionslinjen.
\end{itemize}

\newpage

% Begin full width on the next page (where only projects are shown)
\begin{fullwidth}
\cvsection{Projekt}

\cvevent{Larmsystem}{C, GitHub, Latex}{}{}
\begin{itemize}
  \item I kursen DAT290 var jag projketledare under utvecklingen av ett larmsystem 
  bestående av en mikrodator. Förutom skriva kod i \textbf{C} såg jag till att andra
  gruppmedlemmar viste vad de skulle arbete med och prioritera. Jag höll i veckovisa 
  möten där vi diskuterade framsteg och problem. Resultatet blev bra och jag fick 
  högsta betyg.
  \item Systemet utvecklades i \textbf{C} och \textbf{Github} användes för versionskontroll.
  Dokumentation skrevs i \textbf{Latex}.
\end{itemize}

\divider

\cvevent{\href{https://github.com/fredriknystrom/FIFAWorldCupTip}{VM tips}}{Python, Excel, HTML}{}{}
\begin{itemize}
  \item Första delen av projektet genererar ett automatiserat excel ark skapat i \textbf{Python} 
  och jag använde mig av \textbf{openpyxl} biblioteket. Det genererade arket innehåller bland annat 
  gruppspelsmatcher där användaren sedan fyller i resultatet för varje match. Gruppvinnarna går sedan
  automatiskt vidare till slutspelat och så vidare.
  \item Den andra delen av projektet är ett rättningsskript som läser in excel filer och jämför 
  dem mot en fil med facit. Poängen för varje tip samlas sedan i en text fil.
  \item Sista delen av projektet är en enkel hemsida som visar resultattavan för deltagarna i tipset.
  Hemsidan är skapad i \textbf{HTML} och \textbf{CSS}.
\end{itemize}

% Other merits
\cvsection{Övriga meriter}

\cvevent{Datateknologsektionens idrottskommitté, iDrott}{}{maj 2022 -- pågående}{}
\begin{itemize}
  \item Jag är aktiv i planering och arrangemang av veckovisa idrottsträningar och uppmuntrar
   datateknologer till fysisk aktivitet. 
  \item Jag är lagledare för datateknologsektionens fotbollslag.
\end{itemize}

\divider

\cvevent{Ordförande Brf Volrat Thams styrelse}{}{maj 2022 -- pågående}{}
\begin{itemize}
  \item Arbetet som ordförande innefattar bland annat utskick av kallelse och dagordning till 
  styrelesemöte varje månad samt attestering av leverantörsfakturor. Under möten diskuteras och 
  beslutas omläggning av lån, godkännande av nya medlemmar, reparation och underhåll av fastigheten. 
\end{itemize}


\divider

\cvevent{Svensk Klassiker}{}{2022/2023}{}

\divider

\cvevent{Adlerbertska Stiftelsernas stipendium}{}{2021, 2022}{}

\divider

\cvevent{B-Körtkort, truckkort A2 och A4}{}{2019}{}

% References moved from page 1
\cvsection{Referenser}
Referenser/intyg/betyg medtages vid intervjutillfälle eller skickas in vid förfrågan.

% End fullwidth for second page
\end{fullwidth}

\end{document}
